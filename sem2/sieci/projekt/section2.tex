
\section{Topology description}


\subsection{Mesh}

The goal of this project was to simulate a simple mesh network. The
assumption for all following simulations is as follows: $n$ Nodes
are directly connected physically together and form a simple network.
The physical connection types being simulated are:
\begin{itemize}
\item \textit{Point-To-Point}: This connection type was chosen because of
its simplicity. It suited well first experiments for setting up the
simulated network. Another consideration was the fact that when more
than two nodes are connected, then the left-most and right-most nodes
are not physically connected (see figure \ref{fig:Point-To-Point-connection-between})
which allowed to experiment with the routing capabilities of NS-3.
\item \textit{Wifi Adhoc}: This connection type was choosen for the final
simulation and for the visioned visualization.
\end{itemize}
%
\begin{figure}
\begin{centering}
% Define two helper counters
\begin{tikzpicture}[auto]

   % % grid
   % \def\supertiny{ \font\supertinyfont = cmr9 at 3pt \relax \supertinyfont}
   % \newcounter{gridrows}
   % \setcounter{gridrows}{12}
   % \newcounter{gridcols}
   % \setcounter{gridcols}{30}
   % \draw [gray, very thin] (0, -\arabic{gridrows}) grid (\arabic{gridcols}, 0);
   % \foreach \x in {0,...,\arabic{gridcols}}
   %     \foreach \y in {0,...,\arabic{gridrows}}
   %     {
   %         \draw (\x+0.15, -\y-0.15) node [gray, very thin] {\supertiny{\x/\y}};
   %     }

    % styles
    \tikzstyle{netnode} = [shape=rectangle, draw, rounded corners=2pt, fill=black!20, minimum height=10mm, minimum width=30mm];
    \tikzstyle{darrow} = [latex-latex];

    \draw 
        node[netnode] (node1) {Node 1}
        node[netnode, right=of node1, xshift=2cm] (node2) {Node 2}
        node[netnode, right=of node2, xshift=2cm] (node3) {Node 3};

    \path
        (node1) edge[darrow] node{Point-To-Point} (node2)
        (node2) edge[darrow] node{Point-To-Point} (node3);
\end{tikzpicture}

\par\end{centering}

\caption{\label{fig:Point-To-Point-connection-between}Point-To-Point connection
between three nodes}

\end{figure}



\subsection{Protocol stack}


\subsubsection{TCP/IP}

The protocol stack being used in all following simulations is standard
TCP/IPv4. Actually it is currently the only protocol stack being supported
by NS-3 at least for the higher OSI levels.


\subsubsection{Routing}

Because the current project emphasizes on the visualization of packets
traveling through a real network, the routing of the packets needs
special attention. In NS-3 currently two rouing models are being supported:
\begin{itemize}
\item \textit{Global Routing}: A static routing table being available at
each node of the network. This table is filled by the NS-3 during
the startup of the target simulation. Obviously this routing method
is only applicable for static, non-moving nodes.
\item \textit{OLSR}\cite{rfc3626}: The Optimized Link State Routing Protocol
is an IP rouing protocol for Adhoc mobile network. This protocol is
being used at the OLPC project and is one of currently available routing
protocols.
\end{itemize}

