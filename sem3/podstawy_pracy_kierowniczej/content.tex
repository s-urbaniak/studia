Mała firma jest pojęciem często używanym a nawet modnym. Wiele się mówi  i pisze o roli tej grupy firm w gospodarce polskiej i europejskiej. Pojęcie to jest definiowane w różny sposób, ale najczęściej przyjmowane jest kryterium liczby zatrudnienia.

Komisja Europejska definiuje mała firmę jako tę, która zatrudnia średnio od dziesięciu do pięćdziesięciu osób, jej roczne obroty lub roczna suma bilansowa jest mniejsza niż 10 mln. Euro, oraz nie ma innych małych udziałowców powyżej 25 proc.

Dla potrzeb statystyki GUS definiuje małe przedsiębiorstwo w zależności od działu gospodarki. W przemyśle i budownictwie małe przedsiębiorstwo to takie, które zatrudnia od sześciu do pięćdziesięciu pracowników, w pozostałych działach gospodarki od sześciu do dwudziestu osób. Kryteria zatrudnienia są róże w zależności od kraju. W większości krajów europejskich granica to dwadzieścia osób zatrudnionych za wyjątkiem Niemiec (50), Francji (10). Szwajcaria i Belgia nie definiują małej firmy. W USA mała firma to taka, która zatrudnia do 100 osób.

Definicji małej firmy jest tyle, ilu badaczy i kryteriów analizy. Kryterium ilości jest nieprecyzyjne. Firma produkująca np. meble zatrudniająca 50 osób będzie firmą mała. Natomiast przychodnia lekarska lub biuro rachunkowe – bardzo dużą. Na tym przykładzie widać jak nieprecyzyjne jest to kryterium. Należy wziąć pod uwagę także warunki ekonomiczne polskich firm, gdzie często osoby zatrudnione formalnie na umowę o pracę to tylko część rzeczywiście pracujących w  przedsiębiorstwie. Znaczący zasób ludzki stanowią osoby współpracujące z organizacją i pełniące ważną rolę w strukturze działania. One także powinny być uwzględniane.\footnote{\cite{msr}: s. 9-11}

Obowiązująca w Polsce definicja małego przedsiębiorstwa reguluje Ustawa o swobodzie działalności gospodarczej, która weszła w życie 21 sierpnia 2004 roku.
Zgodnie z przepisami, mały przedsiębiorca to taki, który w co najmniej jednym z dwóch ostatnich lat oborowych zatrudniał średniorocznie mniej niż 50 pracowników oraz osiągnął roczny obrót netto ze sprzedaży towarów, wyrobów i usług oraz operacji finansowych nieprzekraczający równowartości w złotych 10 mln euro lub sumy aktywów jego bilansu sporządzonego na koniec jednego z tych lat nie przekroczyły równowartości powyższej kwoty.\footnote{\cite{paiz}: http://www.paiy.gov.pl}

Sposób zarządzania w malej firmie jest zależny od specyfiki jej działania. Można jednak wyróżnić cechy, które charakteryzują procesy zarządzania zachodzące w tej grupie przedsiębiorstw.
Można powiedzieć, że system zarządzania w małej firmie jest:\footnote{\cite{msr}: s. 18}
\begin{itemize}
\item Prosty – stosuje się uproszczone techniki u narzędzia. Kierownicy często nie mają czasu, a także i wiedzy aby wdrażać systemy zarządzania pracownikami i co więcej nie odczuwają takiej potrzeby.
\item Scentralizowany – decyzje dotyczące personelu podejmowane są jednoosobowo to na najwyższym szczeblu. Rekrutacja i selekcja pracowników to często czytanie nadesłanych życiorysów i listów motywacyjnych przez szefa i podejmowanie decyzji.
Okazuje się, że funkcje personalne delegowane są przez szefów małych firm najpóźniej i z wielkimi oporami. Jest to jedna y cech charakterystycznych małych firm i często ogromna bariera rozwoju. Tworzy się wąskie gardło firmy w osobie szefa.
\item Niesformalizowany – uregulowania formalne wynikają jedynie z przepisów prawa. Opracowane na piśmie standardy i próby ich wprowadzenia często traktuje się jak przejaw biurokracji i niepotrzebne ograniczenie.
Szeroki zakres obowiązków każdego pracownika. Poszczególni pracownicy pełnią często złożony zakres zadań. Ponieważ nie występuje podział na piony, często jeden pracownik realizuje kilka zbliżonych do siebie działań. np.: Sekretarka  to nierzadko pracownik księgowy lub zaopatrzeniowiec. Handlowiec zajmuje się również marketingiem i rozpatruje reklamacje. Prowadzi to często przeciążenia obowiązkami i braku specjalizacji, a co za tym idzie do braku profesjonalizmu. Wywołuje to różne oceny i opinie o pracowniku, często niesprawiedliwe.
Ma to też swoje dobre strony. Pozwala na szybkie reagowanie i dostosowywanie się do zmieniających się warunków na rynku, oraz daje pracownikowi szansę na wszechstronny rozwój.
\end{itemize}

W małych firmach, zwłaszcza rodzinnych, dominuje autokratyczny sposób zarządzania, nieformalne kanały informacji, brak delegowania uprawnień, ręczne zarządzanie operacyjne,  a także nieprecyzyjny i szeroki zakres obowiązków pracowników. Przedsiębiorstwa, rozwijające się z bardzo małych, często rodzinnych interesów, nie nadążają z dostosowaniem metod działania do swojej wielkości i zmieniających się warunków np. znacznego wzrostu ilości pracowników.

Jednak w małej firmie nie ma potrzeby zbytniego formalizowania i biurokratyzowania współpracy, głównie ze względu na koszty. Dopóki zarząd, a często jest to właściciel, panuje nad nieformalnym systemem, zmiana nie jest konieczna. Problem polega na tym, że trudno jest ten moment zaobserwować. Firma zaczyna być zarządzana niesprawnie, traci klientów a co za tym idzie popada w kłopoty. Posiadając doskonałe zaplecze materialne, technologiczne i kapitałowe nie podejmuje działań w zakresie zarządzania zasobami ludzkimi traci swoją przewagę na rynku.

Trzeba powiedzieć, że w małych przedsiębiorstwach często nie jest doceniana rola pracowników jako zasobu budującego trwałą przewagę konkurencyjną. Często właściciele firm skupiają się na rozwoju technologii, budują zaplecze materialne i kapitałowe, traktując zasoby ludzkie jako zawsze dostępne, a tym samym łatwe do uzyskania.\footnote{\cite{ipis}: http://www.ipis.pl}

Szefowie zaczynają uświadamiać sobie, że pomijanie problemów zarządzania w sferze pracowniczej, nie jest możliwe. Brak profesjonalizmu w tej dziedzinie bierze się z braku środków finansowych, wiedzy, oraz oraz kadr wyszkolonych w tym zakresie.

Zarządzanie małym przedsiębiorstwem we wszystkich jego dziedzinach ma charakter intuicyjny i opiera się na jednej osobie. Często jest to właściciel lub osoba zarządzająca (dyrektor, prezes). Styl kierowania wynika z osobowości osoby stojącej na czele przedsiębiorstwa. Można wyróżnić dwa podstawowe style kierowania:

\begin{itemize}
\item Autokratyczny (dyrektywny) charakteryzujący się dystansem do pracowników, arbitralnością w wydawaniu poleceń, nieliczeniem się z opiniami pracowników. Ten styl zarządzania ma swoje zalety i sprawdza się w sytuacjach często występujących w małych firmach: wymagających dużej i szybkiej mobilizacji i ukierunkowania na realizację zadań. Pozwala to dobrze działać w sytuacjach typu nikt nic nie wie oraz wszystko wiadomo. Wadą tego stylu jest to, że pracownicy skupiają się na unikaniu podpadania szefowi, a nie na realizacji zadań. Maleje też poziom  zaangażowania i entuzjazmu do pracy.

\item Partycypacyjny (demokratyczny) opiera się na zasadach współpracy, dążeniu do wspólnego rozwiązywania problemów firmy, włączania pracowników w procesy decyzyjne. Ten sposób zarządzania sprzyja stabilizacji zatrudnienia oraz rozwojowi pracowników. Jest on przydatny w działaniach łączących elementy niejasności i zorganizowanej struktury.
\end{itemize}

Styl partycypacyjny jest ogólnie uznawany za lepszy, choć zawsze trzeba uwzględnić rodzaj firmy jaką się kieruje i szereg innych uwarunkowań. Ważnymi czynnikami determinującymi styl zarządzania są wielkość przedsiębiorstwa, forma własności, dostępność zasobów i funkcja właściciela.\footnote{\cite{msr}: s.20-23}

Kolejnym tematem ważnym dla właścicieli małych firm jest kwestia strategii. Często nie widzą oni potrzeby tworzenia strategii dla swojej firmy. Uważają że świetnie sobie radzą, rozwijają się i nie mają na nią środków.

\emph{Strategia to przyjęta przez kierownictwo organizacji spójna koncepcja działania, której wdrożenie ma zapewnić ciągnięcie fundamentalnych celów długookresowych w ramach wybranej domeny działania.}\footnote{\cite{msr}: s.23}

Należy zadać pytanie czy strategia jest niezbędna dla działania firmy. Zależy ona od etapu jej rozwoju. Często firma jest zakładana bez zamiaru jej rozwijania, właściciel nie wiąże z nią dalekosiężnych planów, powstaje niejako przy okazji. Z czasem pojawiają się nowe okoliczności. Często to przypadek decyduje o jej wzroście. W pewnym momencie brak strategii może stać się barierą dla rozwoju firmy a nawet doprowadzić do jej upadku.

Istotna jest relacja polityki personalnej w strategii organizacji. Pracownicy są jednym z najważniejszych zasobów przedsiębiorstwa. Dlatego strategia powinna być zintegrowana z zasobami ludzkimi. Tu warto przywołać model czterech C:\footnote{\cite{msr}: s.26}
\begin{itemize}
\item kompetencje (competencies)
\item zaangażowanie (commitment)
\item zgodność (congruence)
\item efektywność kosztowa (cost effecttivenss).
\end{itemize}

Współczesna mała firma działa na bardzo konkurencyjnym rynku globalnym, w oparciu o zaawansowane technologie, często w sektorze zaawansowanych usług. Wiedza fachowa (know-how) jest czynnikiem budowania przewagi konkurencyjnej i zasoby ludzkie to często podstawowy kapitał organizacji. Tworzenie strategii organizacji a w tym strategii zarządzania zasobami ludzkimi w małej firmie jest ważnym elementem dla jej rozwoju i funkcjonowania na rynku.
