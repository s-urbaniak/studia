\documentclass[pointlessnumbers, abstracton, headsepline, a4paper]{scrartcl}

\usepackage[T1]{fontenc}
\usepackage[utf8]{inputenc}
\usepackage{graphicx}
\usepackage{microtype}
\usepackage{textcomp}
\usepackage{ellipsis, fixltx2e, mparhack, booktabs, longtable}
\usepackage[automark]{scrpage2}
\usepackage{multicol}
\usepackage{microtype}
\usepackage{listings}
\usepackage[a4paper]{geometry}
\usepackage[polish]{babel}

\usepackage{courier}
\lstset{
         basicstyle=\footnotesize\ttfamily, % Standardschrift
         %numbers=left,               % Ort der Zeilennummern
         numberstyle=\tiny,          % Stil der Zeilennummern
         %stepnumber=2,               % Abstand zwischen den Zeilennummern
         numbersep=5pt,              % Abstand der Nummern zum Text
         tabsize=2,                  % Groesse von Tabs
         extendedchars=true,         %
         breaklines=true,            % Zeilen werden Umgebrochen
         keywordstyle=\color{red},
         stringstyle=\color{white}\ttfamily, % Farbe der String
         showspaces=false,           % Leerzeichen anzeigen ?
         showtabs=false,             % Tabs anzeigen ?
         showstringspaces=false      % Leerzeichen in Strings anzeigen ?        
}

% part of the hyperref bundle
\usepackage{ifpdf}

\geometry{verbose,tmargin=3.5cm,bmargin=3.5cm}
\setlength{\parskip}{\medskipamount}
\setlength{\parindent}{0pt}

\clearscrheadfoot
\ohead{\\\headmark}
\ihead{\includegraphics[scale=0.2]{img/zut2.jpg}}
\ofoot[\pagemark]{\pagemark}

% if pdflatex is used
\ifpdf

%set fonts for nicer pdf view
\IfFileExists{lmodern.sty}{\usepackage{lmodern}}
  {\usepackage[scaled=0.92]{helvet}
    \usepackage{mathptmx}
    \usepackage{courier} }
\fi

% the pages of the TOC are numbered roman
% and a pdf-bookmark for the TOC is added
\pagenumbering{arabic}
\let\myTOC\tableofcontents
\renewcommand\tableofcontents{\myTOC\clearpage\pagenumbering{arabic}}

\begin{document}
\begin{titlepage}

\begin{center}
\includegraphics[scale=0.5]{logos/zut.jpg}
\par
\end{center}

\begin{center}
\textsf{\textbf{\LARGE Wydział Informatyki}}
\end{center}{\LARGE}

\vspace{1.5cm}

\begin{center}
\textsf{\Large Metody sztucznej inteligencji}
\end{center}

\begin{center}
\textsf{\textbf{\Large Laboratorium 01 IUz-22 Urbaniak}}
\end{center}

\begin{center}
\textsf{\large Sprawozdanie}
\end{center}

\vspace{3.5cm}

\begin{center}
\begin{tabular}{ll}
Autor: & Sergiusz Urbaniak\tabularnewline
Grupa: & IUz-22\tabularnewline
Data: & \today\tabularnewline
\end{tabular}
\end{center}

\end{titlepage}

\section{Uczenie neuronu bramek logicznych}

\subsection{Bramka OR}

Bramka OR posiada tablice prawdy pokazaną w tablicy \ref{tab:or}. . Wartośći wejśćowe są wpisane do zmiennej \texttt{we}. Dane wyjśćowe według tablicy prawdy \ref{tab:or} są wpisane w zmienną \texttt{wy}. Następnie jest tworzona ``sieć'' jednego perceptronu i przeprowadzona symulacja działania sieci.

\begin{table}
\begin{center}
\begin{tabular}[t]{c|c}
Wejście & Wyjście \\
\hline
0 0 & 0 \\
0 1 & 1 \\
1 0 & 1 \\
1 1 & 1 \\
\end{tabular}
\end{center}
\caption{\label{tab:or}Bramka OR}
\end{table}

\begin{lstlisting}[caption={uczenie_or.m}, label=lst:bramka_or]
we = [0 0 1 1; 0 1 0 1];
wy = [0 1 1 1];

net = newp(minmax(we), 1);
y = sim(net, we);

figure(1)
plot(abs(y-wy));
\end{lstlisting}

\subsection{Bramka XOR}

\begin{figure}
\begin{center}
\includegraphics[scale=0.6]{logos/zut2}
\par
\end{center}
\caption{\label{fig:OLPC-network-visualization}OLPC network visualization}
\end{figure}

\begin{itemize}
\item C++
\item Python
\end{itemize}

\end{document}

